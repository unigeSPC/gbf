\documentclass[10pt,a4paper]{article}
\usepackage[utf8]{inputenc}
%\usepackage[latin1]{fontenc}
\usepackage{amsmath}
\usepackage{amsfonts}
\usepackage{amssymb}
\usepackage{makeidx}
\usepackage{graphicx}
\usepackage[linktocpage=true]{hyperref}
\usepackage{natbib}
\usepackage[top=1in, bottom=1in, left=1in, right=1in]{geometry}
\usepackage{booktabs}
\usepackage{tabularx}
\usepackage{rotating}
\usepackage{float}
\usepackage{wrapfig}
\usepackage{xcolor}
\usepackage{fancyhdr}
\usepackage{titlesec}
\usepackage{longtable}
\usepackage{booktabs}
\usepackage{enumitem}
\usepackage{hyperref}
\definecolor{titlepagecolor}{cmyk}{0,0,0,0}
\definecolor{k}{cmyk}{1,1,1,1} 
\definecolor{namecolor}{cmyk}{1,0,0.56,.18}  			% Declare here the emphasis color
\usepackage{sidecap} 													% To add caption next to the figure
\usepackage[labelfont=bf, font=small]{caption} 	% Customize captions
\DeclareCaptionFont{namecolor}{\color{namecolor}}
\captionsetup{labelfont={namecolor,bf}}

% Customize section labels
\titleformat{\section}
  {\normalfont\sffamily\huge\bfseries\color{namecolor}}
 {\thesection}{1em}{}
\titleformat{\subsection}
  {\normalfont\sffamily\Large\bfseries\color{namecolor}}
  {\thesubsection}{1em}{}
\titleformat{\subsubsection}
  {\normalfont\sffamily\large\bfseries\color{namecolor}}
  {\thesubsubsection}{1em}{}

\pagestyle{fancy}

\lhead{}
\chead{}
\rhead{}

\lfoot{\color{namecolor}\textsf{GBF User Manual}}
\cfoot{}
\rfoot{\color{namecolor}\textsf{\thepage}}

\renewcommand{\headrulewidth}{0.4pt}
\renewcommand{\footrulewidth}{0.4pt}


\author{Sebastien Biass & Jean-Luc Falcone}
\title{Great Balls of Fire}



\begin{document}

%%%%%%%%%%%%%%%%%%%%%%%%%%%%%%%%%%%%
% 1. Title page 

\begin{titlepage}
\newgeometry{left=7.5cm} %defines the geometry for the titlepage
\pagecolor{titlepagecolor}
\noindent
\includegraphics[width=5cm]{logo.png}\\[2em]  	% Include your custom logo here
\color{namecolor}
\makebox[0pt][l]{\rule{1.3\textwidth}{1pt}}
\par
\noindent
\\[4em] 
\textbf{\textsf{\Huge{Great Balls of Fire}}}
\\[4em] 
\color{k}

\begin{large}
\noindent\textsf{Sébastien Biasse$^1$}\\
\textsf{Jean-Luc Falcone$^2$}\\
\end{large}


\begin{small}
\begin{enumerate}
\item \textit{\textsf{Earth Science, University of Geneva}}
\item \textit{\textsf{Computer Science, University of Geneva}}
\end{enumerate}
\end{small}

\color{namecolor}
\vfill
\noindent
{\huge \textsf{User manual}}
\vskip\baselineskip
\noindent
\color{k}
\textsf{November 2015}
\end{titlepage}
\restoregeometry % restores the geometry
\nopagecolor% Use this to restore the color pages to white


%%%%%%%%%%%%%%%%%%%%%%%%%%%%%%%%%%%%
% 2. Front matter, i.e. page numbering as lowercase roman lettes

\pagenumbering{roman}

\tableofcontents
\cleardoublepage
\listoffigures
\addcontentsline{toc}{section}{List of Figures}
%\cleardoublepage
\listoftables
\addcontentsline{toc}{section}{List of Tables}
\cleardoublepage

% In order to add a section defined in either front or back matter, add the following command under the section definition:
% \addcontentsline{toc}{section}{Name of the section}


  
%%%%%%%%%%%%%%%%%%%%%%%%%%%%%%%%%%%%
% 3. Main text starts here 

\pagenumbering{arabic}

\section{Bomb simulator}

\subsection{Requirements}
\label{sec-1}

GBF is executed on the Java virtual machine (version 8). It is then
compatible with any OS which supports Java, notably GNU/Linux, Apple
MacOSX and MS Windows.

The computation is mainly CPU intensive, each bomb trajectory is
computed independently and results as saved when avaible. Thus, only a
limited amount of memory is required even for large simulations (1GB
should be enough). Multicore CPU or muliprocessor machines will
however greatly speed up the computation.

\subsection{Getting started}
\label{sec-2}

As said above, the only external requirement is the Java virtual machine version 8.
A suitable version can be dowloaded on \href{http://www.oracle.com/technetwork/java/javase/downloads/index.html}{Oracle website}.

\subsubsection{Binary package}
\label{sec-2-1}

GBF is provided as a "fat" jar containing all necessary libraries. It can be dowloaded
directly on \href{http://example.com/download}{the GBF download page}.

You should just copy the \texttt{.jar} file in a work directory, without any
further operations. Similarly, it can be "uninstalled" by deleting the
\texttt{.jar} file.

\subsubsection{Building from source}
\label{sec-2-2}

GBF can be built from the source code. You should install first the
\texttt{sbt} script, instruction are available on the \href{http://www.scala-sbt.org/0.13/tutorial/Manual-Installation.html}{SBT documentation}.

A copy of the source code can be obtained on the \href{http://example}{GBF project page}. If
you use GIT, you can clone the repository using:

\begin{verbatim}
git clone ???????
\end{verbatim}

If you download the source manually, uncompress it into a fresh directory. The
input the following commands:

\begin{verbatim}
$ sbt
> compile
> oneJar
\end{verbatim}

The resulting \texttt{.jar} file will be located in the \texttt{target/scala-2.11/}
directory.

\textbf{NB:} The first time the project is built, all necessary libraries
will be downloaded. This may take several minutes, but successive
build will be much faster.

\subsection{Running GBF}
\label{sec-3}

GBF is a command line only software. To run it you need to first write
a configuration file and the result will be stored in an ouptut file.
Both file syntax are described in the following sections. We recommend
copying the example in ???

To run it:

\begin{verbatim}
java -jar <gbf.jar> <conf file> <numWorkers>
\end{verbatim}

where \texttt{<gbf.jar>} is the Jar file downloaded or built as described in
previous section; \texttt{<conf file>} is the configuration file and
\texttt{<numWorkers>} is the number of threads used in the computation.

\paragraph{Multithread execution}
\label{sec-3-1}

GBF is able to exploit the full power of a multicore/multiprocessor
architecture. In order to reach best performances, the command line
parameter <numWorkers> must equal the number of physical cores in your
machine. Exceeding this number will result in a slow down. It is
advised to disable hyper-threading if possible.

\subsection{Config file}
\label{sec-4}

\subsubsection{Syntax}
\label{sec-4-1}

\paragraph{Types}
\label{sec-4-1-1}

The configuration types are:

\begin{itemize}
\item \textbf{section}: configuration sub-section. See below.
\item \textbf{integer}: integer number (max: 2,147,483,647)
\item \textbf{float}: floating point number (double precision)
\item \textbf{string}: string of characters. Must be double quoted.
\end{itemize}

\paragraph{Sections}
\label{sec-4-1-2}

Configuration parameters are organized into hierarchical sections. Section can be defined
either using braces or by prefixing parameters with section names (separated using a dot).
For instance, both examples below are equivalent:

\begin{verbatim}
#Example 1
terrain {
  demFile = "dem/dem_10m.txt"
  vent {
    E = 496682.0
    N = 4250641.0
    altitude = 400
  }
}

#Example 2
terrain.demFile = "dem/dem_10m.txt"
terrain.vent.E = 496682.0
terrain.vent.N = 4250641.0
terrain.vent.altitude = 400
\end{verbatim}

\paragraph{Include}
\label{sec-4-1-3}

It is possible to add an \texttt{include} directive at the start of a config file refering
to another config file. For instance:

\begin{verbatim}
#In file basic.conf
wind {
  speed = 10
  direction = 12.5
}

#In file highWind.conf
include "basic.conf"
wind.speed = 90
\end{verbatim}

In this example, whenever \texttt{highWind.conf} is used as an input file,
the \texttt{basic.conf} parameters will be used except when redefined in \texttt{highWind.conf}.
Both files are thus equivalent to:

\begin{verbatim}
wind {
  speed = 90
  direction = 12.5
}
\end{verbatim}

\paragraph{System properties}
\label{sec-4-1-4}

To simplify parameter scan such as sensitivity analysis, parameters
can also be redefined as command line options. Relevant parameters
must be prefixed with \texttt{gbf}.

For instance, using bash shell:

\begin{verbatim}
java -Dgbf.wind.speed=40 -jar gbf_0.0.1.jar basic.conf 4
\end{verbatim}

the wind speed will be set to 40 m/s. Command line definitions always
take precedence on configuration files.

\subsubsection{Parameters}
\label{sec-4-2}

Here are the configuration used. Types between parenthesis are defined
above:

\begin{itemize}
\item \texttt{rng}: random generator (section)
\begin{itemize}
\item \texttt{seed}: the random generator seed (integer). Using the same seed,
with the same configuration option, will produce exactly the same
output. Useful for reproducibility.
\end{itemize}
\item \texttt{terrain}: terrain description (section)
\begin{itemize}
\item \texttt{demFile}: DEM terrain file (string). Format ???
\item \texttt{vent}: vent location (section). All following options are in meters.
\begin{itemize}
\item \texttt{E}: vent easting (float)
\item \texttt{N}: vent northing (float)
\item \texttt{altitude}: vent altitude (float)
\end{itemize}
\end{itemize}
\item \texttt{source}: source parameters (section)
\begin{itemize}
\item \texttt{densAvg}: density average in kg/m3 (float)
\item \texttt{densStd}: density standard deviation in kg/m3 (float)
\item \texttt{phiAvg}: size average in $\Phi$ units (float)
\item \texttt{phiStd}: size standard deviation in $\Phi$ units (float)
\item \texttt{velocityAvg}: velocity average in m/s (float)
\item \texttt{velocityStd}: velocity standard deviation in m/s (float)
\item \texttt{inclinationAvg}: inclination average in degree (float). A perfectly vertical
inclination is equal to 0.
\item \texttt{inclinationStd}: inclination standard deviation in degrees (float).
\end{itemize}
\item \texttt{wind}: wind parameters (section)
\begin{itemize}
\item \texttt{speed}: wind speed in m/s (float)
\item \texttt{direction}: wind direction in degrees (float). A direction of 0
corresponds to a wind comming from North.
\end{itemize}
\item \texttt{drag}: parameters used in drag computation (section)
\begin{itemize}
\item \texttt{timeStep}: time step used for trajectory integration in seconds (float)
\item \texttt{pressure0}: pressure at sea level in Pa (float)
\item \texttt{temp0}: temperature at sea level in Kelvins (float)
\item \texttt{thermalLapse}: thermal lapse in ??? (float)
\item \texttt{reducedDragRadius}: radius of the disc area centered on the vent,
where drag is reduced, in meters (float)
\end{itemize}
\item \texttt{experiment}: experiment parameters
\begin{itemize}
\item \texttt{size}: number of bombs in the experiment (integer)
\item \texttt{outputFile}: file name where output is written (string)
\end{itemize}
\end{itemize}

\textbf{Warning:} If you want to save the output file in a different
directory. You should create the directory before running the
simulation.

\subsection{Output}
\label{sec-5}

The output is written to the file defined in the
\texttt{experiment.outputFile} parameter. Every line of these file represents
a different bomb impact. Characteristics of the impact are separated by a single
space. Columns are:

\begin{enumerate}
\item Easting in meters.
\item Northing in meters.
\item Bomb mass in kg
\item Bomb diameter in meters.
\item Kinetic energy at impact in kJ
\item Incidence angle at impact in degrees. A perfectly vertical impact
has an incidence angle of 0.
\item Bomb ejection angle in degrees. A perfectly vertical ejection
has an angle of 0.
\item Bomb flight time in seconds.
\end{enumerate}


\section{FAQ}
\label{sec-6}

\subsection{How to cite ?}
\label{sec-6-1}
\subsection{What language/libraries where used to design GBF ?}
\label{sec-6-2}

GBF was written in the Scala 2.11 language. It uses Akka to
parallelize trajectory computatiob.

\subsection{How to fix the (density, size, ejection speed, ejection angle) ?}
\label{sec-6-3}

Bomb and ejection parameters are defined by providing a average and a
standard deviation. Values will be drawn at random for each bomb. A parameter
can be made constant by setting the standard deviation to 0.

\subsection{How to perform a sensitivity analysis ?}
\label{sec-6-4}

You can override any parameter in the command line. You can then write
a script to launch several execution. For instance if you want to try
several wind speed values using bash, you can use the followin snippet:

\begin{verbatim}
wind="0 10 25 50 100"
for w in wind; do
  out="impacts_$w.dat"
  java -Dgbf.wind.speed=$w -Dgbf.experiment.outputFile=$out \
         -jar gbf.jar default.conf 4
done
\end{verbatim}

Pay attention to use a different output files (provided here by the
\texttt{-Dgbf.experiment.outputFile} option.

\subsection{How to run bigger simulations ?}
\label{sec-6-5}

Since the number of bombs launched in a simulation is limited by the maximum
size of an integer, you can run bigger simulation by launching several
execution with a different seed and a different output file. You can then 
pool the result by concatening the output files. On unix (Linux or MacOSX):

\begin{verbatim}
$ cat output1.dat output2.dat output3.dat > pooled_ouput.dat
\end{verbatim}

\subsection{Do you support distributed memory parallelism ?}
\label{sec-6-6}

No. Only shared memory is currently supported. However you can launch 
several jobs on several nodes (using a different seed each time). The results
can be pooled as described in previous answer.

\subsection{How to request for feature or report a bug ?}
\label{sec-6-7}

You can use our \href{http://www.example.com/bugs}{issue tracker}. However, because our time is unfotunately limited
we will mainly focus on correcting bug and we may be reluctant to add new features.

\subsection{How may I request for help ?}
\label{sec-6-8}

Well, this guide should be enough. If a section is not clear, feel free to
report an issue as described in the previous answer. We will try to
improve its content whenever possible.

We do not plan to provide individual help, except in the context of a
scientific collaboration. Feel free to contact the paper authors if
you are interested in such collaboration.

\subsection{Do you accept contribution ?}
\label{sec-6-9}

Of course. If you plan to collaborate, feel free to fork the project
and send us a pull request with your modifications.

\subsection{What is the GBF license ?}
\label{sec-6-10}

GBF is free and open source software. It is licensed with the GPL 3
license. Refer to the provided \texttt{LICENSE.txt} file for the exact
details. In layman terms you can:

\begin{itemize}
\item Freely download and use the software and its source code.
\item Redistribute the software, under the condition you keep the same license and
you redistribute the source code.
\item Modify the software or integrate it in another project. However if
you plan to redistribute the result, you should license it under a
compatible license.
\end{itemize}

However not covered by GPL, we ask you to cite the original paper if
you publish data generated with GBF.
% Emacs 24.4.1 (Org mode 8.2.10)
\end{document}
